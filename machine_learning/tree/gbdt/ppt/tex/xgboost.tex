% -*- coding: utf-8 -*-

\section{XGBoost}
% http://nbviewer.jupyter.org/github/facaiy/book_notes/blob/master/machine_learning/tree/gbdt/xgboost/intro.ipynb
\subsection{思路来源}
% 决策树:分裂点、叶子值
% 区别
%% tree_boost: loss -> leaf value
%% xgboost: loss -> split critor
%% 反代b_j回loss
\begin{frame}
    J-节点树:
    \begin{equation*}
        f(x) = h \left (x; \{b_j, R_j\}^J_1 \right ) = \displaystyle \sum_{j=1}^J b_j \mathbf{1}(x \in R_j)
    \end{equation*}

    tree ensemble model:
    \begin{equation*}
        \hat{y}_i = \phi(x_i) = \displaystyle \sum_{k=1}^K f_k(x_i)
    \end{equation*}

    GBDT,毎次迭代可描述成最优问题:
    \begin{align*}
        f_m &= \operatorname{arg \, min}_f \sum_{i=1}^n L \left ( y_i, \hat{y}_i + f(x_i) \right ) \\
            &= \operatorname{arg \, min} \mathcal{L}(f)
    \end{align*}
\end{frame}


\subsection{具体推导}
%简单推导
\begin{frame}{正则}
    \begin{align*}
        \mathcal{L}(f) &= \displaystyle \sum_{i=1}^{n} L(y_i, \hat{y}_i + f(x_i)) + \Omega(f) \\[2ex]
        \Omega(f) &= \gamma \|R_j\| + \frac{1}{2} \lambda \|b_j\|^2 \\
        f(x) &= \sum_{j=1}^J b_j \mathbf{1}(x \in R_j)
    \end{align*}
\end{frame}

%说明
\begin{frame}{泰勒展开}
    \begin{align*}
        \mathcal{L}(f) &\approx \sum_{i=1}^n \left [ L(y_i, \hat{y}_i) + g_i f(x_i) + \frac{1}{2} h_i f^2(x_i) \right ] + \Omega(f) \\
        g_i &= \frac{\partial \, L(y_i, \hat{y}_i)}{\partial \hat{y}_i} \\
        h_i &= \frac{\partial^2 \, L(y_i, \hat{y}_i)}{\partial \hat{y}^2_i} \\
    \end{align*}

    \begin{equation*}
        \mathcal{L}(f) = \sum_{j=1}^J \left ( ( \sum_{i \in I_j} g_i ) b_j + \frac{1}{2} (\sum_{i \in I_j} h_i + \lambda) b_j^2 \right ) + \gamma \|R_j\|
    \end{equation*}
    {\tiny
    详细推导可见:\href{http://nbviewer.jupyter.org/github/facaiy/book_notes/blob/master/machine_learning/tree/gbdt/xgboost/intro.ipynb}{xgboost的基本原理与实现简介,颜发才}}
\end{frame}

\begin{frame}{叶子值}
    \begin{align*}
        b_j &= \operatorname{arg \, min}_{b_j} \mathcal{L} \\
            &= \operatorname{arg \, min}_{b_j} \sum_{j=1}^J \left ( ( \sum_{i \in I_j} g_i ) b_j + \frac{1}{2} (\sum_{i \in I_j} h_i + \lambda) b_j^2 \right ) + \gamma \|R_j\|  \\
            &= \sum_{j=1}^J \operatorname{arg \, min}_{b_j} \left ( ( \sum_{i \in I_j} g_i ) \textcolor{red}{b_j} + \frac{1}{2} (\sum_{i \in I_j} h_i + \lambda) \textcolor{red}{b_j^2} \right ) + \gamma \|R_j\|
    \end{align*}

    \vfill

    \begin{equation*}
        b^*_j = - \frac{\sum_{i \in I_j} g_i}{\sum_{i \in I_j} h_i + \lambda}
    \end{equation*}
\end{frame}

\begin{frame}{树生成的分割依据}
    \begin{align*}
        \mathcal{L} &= - \frac{1}{2} \sum_{j=1}^J \frac{(\sum_{i \in I_j} g_i)^2}{\sum_{i \in I_j} h_i + \lambda} + \gamma \|R_j\| \\
                    &= - \frac{1}{2} H + \gamma T
    \end{align*}

    \vfill

    \begin{align*}
        \mathcal{L}_{\text{split}} &= \mathcal{L} - \mathcal{L}_L - \mathcal{L}_R \\
                                  &= \frac{1}{2} (H_L + H_R - H) + \gamma (T - T_L - T_R) \\
                                  &= \frac{1}{2} (H_L + H_R - H) - \gamma
    \end{align*}
\end{frame}

\begin{frame}[fragile]
    最终,树生成公式:
    \begin{align*}
        \mathcal{L} &= - \frac{1}{2} H + \gamma T \\
        b^*_j &= - \frac{\sum_{i \in I_j} g_i}{\sum_{i \in I_j} h_i + \lambda}
    \end{align*}

    \begin{lstlisting}[language=C++,style=myScalastyle]
void UpdateOneIter(int iter, DMatrix* train) override {
    CHECK(ModelInitialized())
        << "Always call InitModel or LoadModel before update";
    if (tparam.seed_per_iteration || rabit::IsDistributed()) {
        common::GlobalRandom().seed(tparam.seed * kRandSeedMagic + iter);
    }
    this->LazyInitDMatrix(train);
    this->PredictRaw(train, &preds_);
    obj_->GetGradient(preds_, train->info(), iter, &gpair_);
    gbm_->DoBoost(train, this->FindBufferOffset(train), &gpair_);
}
    \end{lstlisting}
    {\tiny \tt
    source: xgboost/src/learner.cc \\[-2ex]
    commit: 49bdb5c97fccd81b1fdf032eab4599a065c6c4f6}
\end{frame}


\subsection{重要参数}
% parameter table

% 选重要的几个来说明
% https://github.com/dmlc/xgboost/blob/master/doc/how_to/param_tuning.md
% https://github.com/dmlc/xgboost/blob/master/doc/parameter.md
% https://www.analyticsvidhya.com/blog/2016/03/complete-guide-parameter-tuning-xgboost-with-codes-python/

%% 注意:这应是弱分类器
\begin{frame}
    \begin{itemize}
        \item \href{https://github.com/dmlc/xgboost/blob/master/doc/parameter.md}{XGBoost Parameters}
            \begin{itemize}
                \item objective \\
                      reg:linear, binary:logistic, multi:softmax
                \item num\_round, max\_depth
                \item eta
                \item lambda (L2 reg), alpha (L1 reg)
            \end{itemize}

        \item \href{https://github.com/dmlc/xgboost/blob/master/doc/how_to/param_tuning.md}{Notes on Parameter Tuning}
    \end{itemize}
\end{frame}
