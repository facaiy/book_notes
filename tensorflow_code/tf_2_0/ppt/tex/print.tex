% -*- coding: utf-8 -*-

\subsection{tf.print}
\begin{frame}{主要变动}
    similar to the standard python print API.\footnote{\href{https://github.com/tensorflow/community/pull/14}{RFC: New tf.print}}


    \begin{itemize}
        \item tf.Print $\to$ tf.print, tf.strings.format
            \begin{itemize}
                \item For python 2: \lstinline{from __future__ import print_function}
            \end{itemize}
        \item identity op $\to$ control dependencies
        \item controllable logging levels
            \begin{itemize}
                \item stdout/sterr,与notebook不兼容
                \item device: cpu:0 by default?
            \end{itemize}
        \item supports for nested data structures
    \end{itemize}
\end{frame}

\begin{frame}[fragile]
%\begin{lstlisting}[language=Python,style=myScalastyle, caption=eager mode]
%tf.enable_eager_execution()
%tensor = tf.range(10)
%tf.print(tensor, output_stream=sys.stderr)
%# (This prints "[0 1 2 ... 7 8 9]" to sys.stderr)
%\end{lstlisting}
%
%\begin{lstlisting}[language=Python,style=myScalastyle, caption=graph mode]
%with sess.as_default():
%  tensor = tf.range(10)
%  print_op = tf.print(tensor, output_stream=sys.stdout)
%  # For tf 1.0: return an identity op:
%  # doubled_tensor = print_op * 2
%  # For tf 2.0:
%  with tf.control_dependencies([print_op]):
%    doubled_tensor = tensor * 2
%  sess.run(doubled_tensor)
%  # (This prints "[0 1 2 ... 7 8 9]" to sys.stdout)
%\end{lstlisting}
\end{frame}
