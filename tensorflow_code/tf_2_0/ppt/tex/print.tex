% -*- coding: utf-8 -*-

\subsection{tf.print}
\begin{frame}{tf.print}
    similar to the standard python print API.\footnote{\href{https://github.com/tensorflow/community/pull/14}{RFC: New tf.print}}

    \begin{itemize}
        \item tf.Print $\to$ tf.print, tf.strings.format
            \begin{itemize}
                \item For python 2: \lstinline{from __future__ import print_function}\footnote{\href{https://python3statement.org/\#sections40-timeline}{Moving to require Python 3}},\footnote{\href{http://python-future.org/compatible\_idioms.html}{Cheat Sheet: Writing Python 2-3 compatible code}}
            \end{itemize}
        \item identity op $\to$ control dependencies
        \item controllable logging levels
            \begin{itemize}
                \item stdout/stderr,与notebook不兼容
                \item device: cpu:0 by default?
            \end{itemize}
        \item supports for nested data structures
    \end{itemize}
\end{frame}

\begin{frame}[fragile]
    \begin{tcblisting}{title=eager mode}
        tf.enable_eager_execution()
        tensor = tf.range(10)
        tf.print(tensor, output_stream=sys.stderr)
        # (This prints "[0 1 2 ... 7 8 9]" to sys.stderr)
    \end{tcblisting}

    \begin{tcblisting}{title=graph mode}
        with sess.as_default():
          tensor = tf.range(10)
          print_op = tf.print(tensor, output_stream=sys.stdout)
          # For tf 1.0: return an identity op:
          # doubled_tensor = print_op * 2
          # For tf 2.0:
          with tf.control_dependencies([print_op]):
            doubled_tensor = tensor * 2
          sess.run(doubled_tensor)
          # (This prints "[0 1 2 ... 7 8 9]" to sys.stdout)
    \end{tcblisting}
\end{frame}
